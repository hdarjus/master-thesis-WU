\section{Empirical framework}

We would like to compare Chinese and German common stock's leverage effect independent of both time and company.
For that reason several stocks and time periods are chosen, and the model is fitted to each independently.
As a by-product, if it exists, the systematic time dependence of the leverage effect can be observed as well.

\subsection{Data}

Two stock indices served as starting points for the two countries: the SSE 50 Index from Shanghai for China, and the DAX Index from Frankfurt for Germany.
Then 10-10 companies were chosen randomly from the 50+30 currently listed ones such that they have a long enough stock price history, dating back to 2004.
Tables~\ref{tab:gercompanies} and~\ref{tab:chicompanies} show the final choice.
\begin{table}[h!]
	\centering
	\begin{tabular}{lr}
		\textbf{Company name} & \textbf{Bloomberg ticker} \\
		\hline
		BASF SE & BAS GY Equity \\
		Bayerische Motoren Werke AG & BMW GY Equity \\
		Commerzbank AG & CBK GY Equity \\
		Deutsche Telekom AG & DTE GY Equity \\
		HeidelbergCement AG & HEI GY Equity \\
		Linde AG & LIN GY Equity \\
		Merck KGaA & MRK GY Equity \\
		\thead[cl]{M\"unchener R\"uckversicherungs-\\\ Gesellschaft AG in M\"unchen} & MUV2 GY Equity \\
		SAP SE & SAP GY Equity \\
		Siemens AG & SIE GY Equity
	\end{tabular}
	\caption{German companies}
	\label{tab:gercompanies}
\end{table}

\begin{table}[h!]
	\centering
	\begin{tabular}{lr}
		\textbf{Company name} & \textbf{Bloomberg ticker} \\
		\hline
		\thead[cl]{Shanghai Pudong Development\\\ Bank Co., Ltd.} & 600000 CH Equity \\
		\thead[cl]{China Minsheng Bank} & 600016 CH Equity \\
		\thead[cl]{Citic Securities Co., Ltd.} & 600030 CH Equity \\
		\thead[cl]{China United Network\\\ Communications Ltd.} & 600050 CH Equity \\
		\thead[cl]{SAIC Motor Co., Ltd.} & 600104 CH Equity \\
		\thead[cl]{China Northern Rare Earth\\\ (Group) High-Tech Co., Ltd} & 600111 CH Equity \\
		\thead[cl]{China Fortune Land\\\ Development Co., Ltd.} & 600340 CH Equity \\
		\thead[cl]{Kweichow Moutai Co., Ltd.} & 600519 CH Equity \\
		\thead[cl]{Haitong Securities Co., Ltd} & 600837 CH Equity \\
		\thead[cl]{Inner Mongolia Yili\\\ Industrial Group Co., Ltd.} & 600887 CH Equity
	\end{tabular}
	\caption{Chinese companies}
	\label{tab:chicompanies}
\end{table}

Altogether 32 periods were used, all 3-year-long, in a moving window with step size of 3 months.
The first period is 2004/01/01-2006/12/31 and the last one 2011/10/01-2014/09/30.
This way the dataset fully includes the Subprime crisis of 2007/08.

\subsection{Setup}

Almost the same prior distributions were used as in Section~\ref{sec:simulation}:
\begin{align*}
\frac{\phi+1}2 &\sim\text{Beta}(20,1.5), \\
\sigma^2 &\sim\text{InverseGamma}(2.25,\text{rate}=0.0625), \\
\rho &\sim\mathcal{U}(-1,1), \\
\mu &\sim\mathcal{N}(-9,1), \\
h_1\mid\phi,\sigma,\mu &\sim\mathcal{N}(\mu,\sigma^2/(1-\phi^2)).
\end{align*}
The only exception is $\sigma^2$, whose prior was copied from~\cite{Omori2007}.
In order to achieve convergence, a burn-in of 50,000 draws were used, and then 100,000 samples were recorded for evaluation, hence, a total of $20\times32\times(100,000+50,000)=96,000,000$ samples were drawn for data set sizes of around 750.
