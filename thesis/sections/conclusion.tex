\section{Conclusion}

The main research question is whether Chinese stocks show a positive return-variance correlation.
We cannot claim that in general since, in all periods, more than half of the Chinese companies' leverage effect is insignificant, i.e.\ the posterior $\rho$'s 5\% to 95\% credible interval contains the 0.
On the other hand, we see that the Chinese market participants uniformly have weaker leverage effect than the German ones.
Moreover, sometimes they even have anti-leverage.

Examining further differences and similarities is most interesting with the global crisis of 2007 in mind.
We found proof for $\rho$ being shifted to the negative direction throughout the crisis.
Even though the negative shift is present unquestionably in both countries, the hectic changes of the leverage effect in China bring doubts about the importance of the crisis as a driving factor.

While $\rho$ behaves similarly in China and in Germany throughout the crisis, estimates of $\phi$ contrast the two countries at the same period.
When there is a trend in the volatility, i.e.\ it is not constant plus white noise, then there is high autocorrelation and persistence.
These trends exist in the German stocks' volatilities around the Subprime Crisis and the European Debt Crisis, so at those times $\phi$ is considerably larger than its prior.
On the contrary, this phenomenon is not present in China.

\subsection*{Future work}

While there are many possible explanations for the leverage effect, there is not any for the anti-leverage effect, to the best of our knowledge.
Unusual regulations are likely to be the main factor here, but a different investment culture is also a possible reason.
Either way, a microeconomics-based approach could be the key that models agents maximising utility in a well designed environment.

Finally, time-varying leverage effect models could be another direction for improvements.
Simple linear functions would not be sufficient to model the shifts in crises, probably a higher order polynomial that still avoids overfitting, or an ARCH process for $\rho$ would increase the predictive power.
We did not find such attempts in the literature.

