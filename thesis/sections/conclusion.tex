\section{Conclusion}

The main research question is whether, based on our observations, Chinese stocks showed a positive return-variance correlation.
That we cannot claim in general since in all periods more than half of the Chinese companies' $\rho$ estimate was not significantly different from 0.
On the other hand, what we can say is that, outside of the deepest times of the 2007 crisis, the Chinese market participants uniformly show weaker leverage effect than the German ones, most of the times an insignificant one, sometimes they even show anti-leverage.

Examining further differences and similarities is most interesting with the global crisis of 2007 in mind.
We found proof for $\rho$ being shifted to the negative direction throughout the crisis contained by the sample.
That behaviour was present unquestionably in both countries, although the hectic changes of the leverage effect in China bring doubts about the importance of the crisis as a driving factor for them.

While $\rho$ does not necessarily do, estimates of $\phi$ separate the two countries.
When there is a trend in the volatility, i.e.\ it is not constant plus white noise, then there is high autocorrelation and persistence.
These trends exist in the German stocks' volatilities around the Subprime Crisis and the European Debt Crisis, at those times $\phi$ is significantly larger than its prior, while this phenomenon is not present in China.

\subsection{Future work}

While there are many possible explanations for the leverage effect, there is not any for the anti-leverage effect, to the best of our knowledge.
Unusual regulations are likely to be the main factor here, or a different investment culture.
Either way, a microeconomics-based approach could be the key that models agents maximising utility in a well designed environment.

Finally, time-varying leverage effect models could be another direction for improvements.
Simple linear functions would not be sufficient to model the shifts in crises, probably a higher order polynomial that still avoids overfitting, or an ARCH process for $\rho$ would increase the predictive power.
We did not find such attempts in the literature.

