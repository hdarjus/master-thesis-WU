\section{Literature review}

aaa

It was first described by~\citet{black1976studies}, who proposed two possible explanations.
The first one, and that is probably where the name comes from, was the connection to firms' leverage ratio, i.e. the market equity to debt ratio.
This theory is underlined by the fact that the existence of the leverage effect can be proved in basic structural models of corporate finance, e.g.~\citeauthor{Christie1982} in the Modigliani--Miller world and some of its generalisations.

The effect has been shown using both realised volatility and implied volatility of various models~\citep{Bouchaud2001,Harvey1996,Christie1982,french1987expected}.

Input of Heston model

\subsubsection{Example from Modiglinani--Miller}

The presence of the phenomenon can also be deduced in widely used structural models of corporate finance, e.g. generalisations of Modigliani--Miller~\citep{Christie1982}.
An illustrating example:  

The leverage effect, definition, illustration, references to its existence, where else is it useful

The leverage effect in Modigliani-Miller and other places

Plots

\subsubsection{Leverage in China}

Leverage in China, references

\subsection{Objective}

Research questions

Possible methods, used method (no formulae)