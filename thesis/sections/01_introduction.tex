\section{Introduction}

Modelling the variance\footnote{In this document variance means variance of the returns and volatility is the standard deviation of returns.} of the rate of return on stocks is a cornerstone of modern finance.
Variance is a basic building block of risk measurement and prediction uncertainty, and capturing empirical facts about its behaviour is the aim of many theorists~\citep{Christie1982}.
A well-known fact is the seasonality of volatility: periods of highs and lows alternate~\citep{schwert1989why}.
This master thesis is concerned with another empirical fact, the so-called leverage effect that captures the correlation between volatility and return.

\subsection{The leverage effect}

The leverage effect is, by definition, the negative relationship between the change in return and the change in volatility.
It was first described by~\citet{black1976studies}, and he connected it with the leverage ratio in corporate finance, the ratio of equity to assets.
The presence of the leverage effect can be deduced in widely used structural models of corporate finance, e.g. generalisations of Modigliani--Miller~\citep{Christie1982}.

The leverage effect, definition, illustration, references to its existence, where else is it useful

The leverage effect in Modigliani-Miller and other places

Plots

Leverage in China, references

\subsection{Objective}

Research questions

Possible methods, used method (no formulae)